\documentclass[a4paper,12pt]{article}
\usepackage{cmap}
\usepackage[T2A]{fontenc}
\usepackage[utf8]{inputenc}
\usepackage[english,russian]{babel}
\usepackage{graphicx}
\graphicspath{C:\Users\ereme\Desktop\LabMod1}
\author{Тиванов Данил}
\title{Отчет по лабораторной работе №1}
\date{\today}
\DeclareGraphicsExtensions{.pdf, .png, .jpg}
\begin{document}

\maketitle 
\section{информация о лабораторной работе}	
Целью данной лабораторной работы является изучение метода статистических испытаний (метода Монте-Карло). Данные методы - это группа численных методов, основанных на воспроизведении большого числа реализаций одного или нескольких случайных процессов. В общем, суть метода
заключается в статистическом моделировании случайного процесса, таким
образом, чем больше испытаний будет проведено, тем точнее получится результат. Основными задачами, решаемыми данными методами, являются:
1. численное интегрирование
2. расчеты в системах массового обслуживания
3. расчет качества и надежности изделий
4. задачи теории игр
5. и др

В рамках данной лабораторной работы вы должны выполнить:
1. Написать алгоритм вычисления площади при помощи метода МонтеКарло;
(a) Задать квадрат случайной величины;
(b) Задать произвольную фигуру внутри квадрата (фигуру создавать как минимум с 6-ти углами);
(c) Провести N испытаний по бросанию точки;
(d) Подсчитать площадь S1 фигуры внутри по формуле (1);
(e) Подсчитать площадь S
theor
1 фигуры внутри аналитически;
(f) Найти разницу площадей .
2. Визуализировать квадрат, фигуру внутри и моделируемые точки (сделать это в виде графика);
3. Вывести график зависимости погрешности вычислений от количества испытаний N (10 испытаний, 50 испытаний, 100, 1000, и т.д.);
4. Написать отчет.

\section{Результаты лабораторной работы}
\begin{figure}
	\centering
	\includegraphics[width=0.7\linewidth]{LabMod1/Screenshot_1}
	\caption{}
	\label{fig:screenshot1}
\end{figure}
\begin{figure}
	\centering
	\includegraphics[width=0.7\linewidth]{LabMod1/Screenshot_2}
	\caption{}
	\label{fig:screenshot2}
\end{figure}
\begin{figure}
	\centering
	\includegraphics[width=0.7\linewidth]{LabMod1/Screenshot_3}
	\caption{}
	\label{fig:screenshot3}
\end{figure}
\begin{figure}
	\centering
	\includegraphics[width=0.7\linewidth]{LabMod1/Screenshot_4}
	\caption{}
	\label{fig:screenshot4}
\end{figure}
\begin{figure}
	\centering
	\includegraphics[width=0.7\linewidth]{LabMod1/Screenshot_5}
	\caption{}
	\label{fig:screenshot5}
\end{figure}
\begin{verbatim}
#include <iostream>
#include <math.h>
#pragma once

namespace LabModelir1 {
	
	using namespace System;
	using namespace System::ComponentModel;
	using namespace System::Collections;
	using namespace System::Windows::Forms;
	using namespace System::Data;
	using namespace System::Drawing;
	
	/// <summary>
	/// Сводка для MyForm
	/// </summary>
	public ref class MyForm : public System::Windows::Forms::Form
	{
		public:
		MyForm(void)
		{
			InitializeComponent();
			//
			//TODO: добавьте код конструктора
			//
		}
		
		protected:
		/// <summary>
		/// Освободить все используемые ресурсы.
		/// </summary>
		~MyForm()
		{
			if (components)
			{
				delete components;
			}
		}
		private: System::Windows::Forms::Button^ button1;
		protected:
		private: System::Windows::Forms::Button^ button2;
		private: System::Windows::Forms::TableLayoutPanel^ tableLayoutPanel1;
		private: System::Windows::Forms::PictureBox^ pictureBox1;
		
		
		private: System::Windows::Forms::Label^ label1;
		private: System::Windows::Forms::Label^ label2;
		private: System::Windows::Forms::Label^ label3;
		private: System::Windows::Forms::Label^ label4;
		
		
		private: System::Windows::Forms::Button^ button3;
		private: System::Windows::Forms::TextBox^ textBox4;
		private: System::Windows::Forms::TextBox^ textBox3;
		private: System::Windows::Forms::TextBox^ textBox2;
		private: System::Windows::Forms::TextBox^ textBox1;
		
		
		protected:
		
		
		
		
		
		
		
		
		protected:
		
		
		
		protected:
		
		
		protected:
		
		private:
		/// <summary>
		/// Обязательная переменная конструктора.
		/// </summary>
		System::ComponentModel::Container^ components;
		
		#pragma region Windows Form Designer generated code
		/// <summary>
		/// Требуемый метод для поддержки конструктора — не изменяйте 
		/// содержимое этого метода с помощью редактора кода.
		/// </summary>
		void InitializeComponent(void)
		{
			this->button1 = (gcnew System::Windows::Forms::Button());
			this->button2 = (gcnew System::Windows::Forms::Button());
			this->tableLayoutPanel1 = (gcnew System::Windows::Forms::TableLayoutPanel());
			this->pictureBox1 = (gcnew System::Windows::Forms::PictureBox());
			this->label1 = (gcnew System::Windows::Forms::Label());
			this->label2 = (gcnew System::Windows::Forms::Label());
			this->label3 = (gcnew System::Windows::Forms::Label());
			this->label4 = (gcnew System::Windows::Forms::Label());
			this->button3 = (gcnew System::Windows::Forms::Button());
			this->textBox4 = (gcnew System::Windows::Forms::TextBox());
			this->textBox3 = (gcnew System::Windows::Forms::TextBox());
			this->textBox2 = (gcnew System::Windows::Forms::TextBox());
			this->textBox1 = (gcnew System::Windows::Forms::TextBox());
			this->tableLayoutPanel1->SuspendLayout();
			(cli::safe_cast<System::ComponentModel::ISupportInitialize^>(this->pictureBox1))->BeginInit();
			this->SuspendLayout();
			// 
			// button1
			// 
			this->button1->Location = System::Drawing::Point(604, 41);
			this->button1->Name = L"button1";
			this->button1->Size = System::Drawing::Size(153, 23);
			this->button1->TabIndex = 0;
			this->button1->Text = L"Фигура";
			this->button1->UseVisualStyleBackColor = true;
			this->button1->Click += gcnew System::EventHandler(this, &MyForm::button1_Click);
			// 
			// button2
			// 
			this->button2->Location = System::Drawing::Point(604, 70);
			this->button2->Name = L"button2";
			this->button2->Size = System::Drawing::Size(153, 23);
			this->button2->TabIndex = 0;
			this->button2->Text = L"Точки";
			this->button2->UseVisualStyleBackColor = true;
			this->button2->Click += gcnew System::EventHandler(this, &MyForm::button2_Click);
			// 
			// tableLayoutPanel1
			// 
			this->tableLayoutPanel1->ColumnCount = 1;
			this->tableLayoutPanel1->ColumnStyles->Add((gcnew System::Windows::Forms::ColumnStyle(System::Windows::Forms::SizeType::Percent,
			100)));
			this->tableLayoutPanel1->ColumnStyles->Add((gcnew System::Windows::Forms::ColumnStyle(System::Windows::Forms::SizeType::Absolute,
			20)));
			this->tableLayoutPanel1->Controls->Add(this->pictureBox1, 0, 0);
			this->tableLayoutPanel1->Location = System::Drawing::Point(1, 2);
			this->tableLayoutPanel1->Name = L"tableLayoutPanel1";
			this->tableLayoutPanel1->RowCount = 1;
			this->tableLayoutPanel1->RowStyles->Add((gcnew System::Windows::Forms::RowStyle(System::Windows::Forms::SizeType::Percent, 100)));
			this->tableLayoutPanel1->RowStyles->Add((gcnew System::Windows::Forms::RowStyle(System::Windows::Forms::SizeType::Absolute, 600)));
			this->tableLayoutPanel1->Size = System::Drawing::Size(600, 600);
			this->tableLayoutPanel1->TabIndex = 1;
			// 
			// pictureBox1
			// 
			this->pictureBox1->Dock = System::Windows::Forms::DockStyle::Fill;
			this->pictureBox1->Location = System::Drawing::Point(3, 3);
			this->pictureBox1->Name = L"pictureBox1";
			this->pictureBox1->Size = System::Drawing::Size(594, 594);
			this->pictureBox1->TabIndex = 0;
			this->pictureBox1->TabStop = false;
			// 
			// label1
			// 
			this->label1->Anchor = System::Windows::Forms::AnchorStyles::Top;
			this->label1->AutoSize = true;
			this->label1->Location = System::Drawing::Point(635, 337);
			this->label1->Name = L"label1";
			this->label1->Size = System::Drawing::Size(60, 13);
			this->label1->TabIndex = 4;
			this->label1->Text = L"Введите N";
			// 
			// label2
			// 
			this->label2->AutoSize = true;
			this->label2->Location = System::Drawing::Point(635, 452);
			this->label2->Name = L"label2";
			this->label2->Size = System::Drawing::Size(88, 13);
			this->label2->TabIndex = 4;
			this->label2->Text = L"S Аналитически";
			// 
			// label3
			// 
			this->label3->AutoSize = true;
			this->label3->Location = System::Drawing::Point(635, 398);
			this->label3->Name = L"label3";
			this->label3->Size = System::Drawing::Size(83, 13);
			this->label3->TabIndex = 5;
			this->label3->Text = L"S Практически";
			// 
			// label4
			// 
			this->label4->AutoSize = true;
			this->label4->Location = System::Drawing::Point(635, 513);
			this->label4->Name = L"label4";
			this->label4->Size = System::Drawing::Size(104, 13);
			this->label4->TabIndex = 6;
			this->label4->Text = L"Разница площадей";
			// 
			// button3
			// 
			this->button3->Location = System::Drawing::Point(604, 99);
			this->button3->Name = L"button3";
			this->button3->Size = System::Drawing::Size(153, 23);
			this->button3->TabIndex = 7;
			this->button3->Text = L"График";
			this->button3->UseVisualStyleBackColor = true;
			this->button3->Click += gcnew System::EventHandler(this, &MyForm::button3_Click);
			// 
			// textBox4
			// 
			this->textBox4->Location = System::Drawing::Point(604, 529);
			this->textBox4->Name = L"textBox4";
			this->textBox4->Size = System::Drawing::Size(153, 20);
			this->textBox4->TabIndex = 3;
			this->textBox4->TextChanged += gcnew System::EventHandler(this, &MyForm::textBox4_TextChanged);
			// 
			// textBox3
			// 
			this->textBox3->Location = System::Drawing::Point(604, 468);
			this->textBox3->Name = L"textBox3";
			this->textBox3->Size = System::Drawing::Size(153, 20);
			this->textBox3->TabIndex = 3;
			this->textBox3->TextChanged += gcnew System::EventHandler(this, &MyForm::textBox3_TextChanged);
			// 
			// textBox2
			// 
			this->textBox2->Location = System::Drawing::Point(604, 414);
			this->textBox2->Name = L"textBox2";
			this->textBox2->Size = System::Drawing::Size(153, 20);
			this->textBox2->TabIndex = 3;
			this->textBox2->TextChanged += gcnew System::EventHandler(this, &MyForm::textBox2_TextChanged);
			// 
			// textBox1
			// 
			this->textBox1->Location = System::Drawing::Point(604, 353);
			this->textBox1->Name = L"textBox1";
			this->textBox1->Size = System::Drawing::Size(153, 20);
			this->textBox1->TabIndex = 2;
			this->textBox1->Text = L"10";
			this->textBox1->TextChanged += gcnew System::EventHandler(this, &MyForm::textBox1_TextChanged);
			// 
			// MyForm
			// 
			this->AutoScaleDimensions = System::Drawing::SizeF(6, 13);
			this->AutoScaleMode = System::Windows::Forms::AutoScaleMode::Font;
			this->ClientSize = System::Drawing::Size(759, 611);
			this->Controls->Add(this->button3);
			this->Controls->Add(this->label4);
			this->Controls->Add(this->label3);
			this->Controls->Add(this->label2);
			this->Controls->Add(this->label1);
			this->Controls->Add(this->textBox4);
			this->Controls->Add(this->textBox3);
			this->Controls->Add(this->textBox2);
			this->Controls->Add(this->textBox1);
			this->Controls->Add(this->tableLayoutPanel1);
			this->Controls->Add(this->button2);
			this->Controls->Add(this->button1);
			this->Name = L"MyForm";
			this->Text = L"MyForm";
			this->tableLayoutPanel1->ResumeLayout(false);
			(cli::safe_cast<System::ComponentModel::ISupportInitialize^>(this->pictureBox1))->EndInit();
			this->ResumeLayout(false);
			this->PerformLayout();
			
		}
		#pragma endregion
		private: System::Void Dr0w() {
			int pw = pictureBox1->Width, ph = pictureBox1->Height;
			Bitmap^ bmp = gcnew Bitmap(pw, ph);
			Graphics^ graph = Graphics::FromImage(bmp);
			graph->DrawRectangle(Pens::Black, 10, 10, 550, 550);
			graph->DrawLine(Pens::Blue, 60, 50, 30, 200);
			graph->DrawLine(Pens::Blue, 30, 200, 150, 440);
			graph->DrawLine(Pens::Blue, 150, 440, 322, 512);
			graph->DrawLine(Pens::Blue, 322, 512, 520, 390);
			graph->DrawLine(Pens::Blue, 520, 390, 490, 100);
			graph->DrawLine(Pens::Blue, 490, 100, 60, 50);
			this->pictureBox1->Image = bmp;
		}
		
		private: int Pl0shad3(int x1, int y1, int x2, int y2, int x3, int y3) {
			long long Str = 1;
			Str = abs((x2 - x1) * (y3 - y1) - (x3 - x1) * (y2 - y1)) / 2;
			return Str;
		}
		//   180, 150, 150, 440, 322, 512
		private: int Pl0shad6(int x, int y) {
			long long Ssh = 0;
			Ssh += Pl0shad3(x, y, 60, 50, 30, 200);
			Ssh += Pl0shad3(x, y, 30, 200, 150, 440);
			Ssh += Pl0shad3(x, y, 150, 440, 322, 512);
			Ssh += Pl0shad3(x, y, 322, 512, 520, 390);
			Ssh += Pl0shad3(x, y, 520, 390, 490, 100);
			Ssh += Pl0shad3(x, y, 490, 100, 60, 50);
			return Ssh;
		}
		
		private: System::Void button1_Click(System::Object^ sender, System::EventArgs^ e) {
			Dr0w();
		}
		
		public:	  long long Area_Theoretically;
		public:	  double Formula_Error;
		public:   long long Amount_Of_Points;
		public:   long long Figure_Area ;
		public:	  long long Square_Area = 540 * 540;
		public:	  long long Area_Difference;
		
		private: System::Void button2_Click(System::Object^ sender, System::EventArgs^ e) {
			int N = int::Parse(textBox1->Text);
			int pw = pictureBox1->Width, ph = pictureBox1->Height;
			Bitmap^ bmp = gcnew Bitmap(pw, ph);
			Graphics^ graph = Graphics::FromImage(bmp);
			graph->DrawRectangle(Pens::Black, 10, 10, 550, 550);
			graph->DrawLine(Pens::Blue, 60, 50, 30, 200);
			graph->DrawLine(Pens::Blue, 30, 200, 150, 440);
			graph->DrawLine(Pens::Blue, 150, 440, 322, 512);
			graph->DrawLine(Pens::Blue, 322, 512, 520, 390);
			graph->DrawLine(Pens::Blue, 520, 390, 490, 100);
			graph->DrawLine(Pens::Blue, 490, 100, 60, 50);
			textBox2->Clear();
			textBox3->Clear();
			textBox4->Clear();
			this->pictureBox1->Image = bmp;
			
			Figure_Area = Pl0shad6(200, 225);
			textBox3->Text += Convert::ToString(Figure_Area);
			Amount_Of_Points = 0;
			for (int i = 0; i < N; i++) {
				int rnd1, rnd2;
				rnd1 = rand() % 550 + 10;
				rnd2 = rand() % 550 + 10;
				if (Figure_Area == Pl0shad6(rnd1, rnd2)) Amount_Of_Points++;
				graph->DrawRectangle(Pens::Red, rnd1, rnd2, 1, 1);
			}
			Area_Theoretically = Amount_Of_Points * Square_Area / N;
			textBox2->Text += Convert::ToString(Area_Theoretically);
			Area_Difference = abs(Area_Theoretically - Figure_Area);
			textBox4->Text += Convert::ToString(Area_Difference);
		}
		
		public: delegate double DelegatePtr(double); 
		private: System::Void button3_Click(System::Object^ sender, System::EventArgs^ e) {
			int pW = pictureBox1->Width, pH = pictureBox1->Height;
			Bitmap^ img = gcnew Bitmap(pW, pH);
			Graphics^ g = Graphics::FromImage(img);
			int mX = int(pW / 2 - pW / 2 % 1);
			int mY = int(pH / 2 - pH / 2 % 1);
			g->DrawLine(Pens::Red, mX, 0, mX, pH);
			g->DrawLine(Pens::Red, 0, mY, pW, mY);
			g->ScaleTransform(1, -1);
			g->TranslateTransform((float)mX, -(float)mY); 
			
			double y; double x; double ctp;
			
			System::Collections::Generic::List<PointF>^ Points =	
			gcnew System::Collections::Generic::List<PointF>(); 
			textBox4->Clear();
			for (int N = 10; N < 1000; N ++ ) {	
				x = N;
				Area_Theoretically = Amount_Of_Points * Square_Area / N;
				Area_Difference = abs(Area_Theoretically - Figure_Area);
				ctp = Area_Difference;
				ctp /= Square_Area;
				y = ctp*100;	
				Points->Add(PointF(x , y ));
				
			}
			
			g->DrawLines(Pens::Green, Points->ToArray()); 
			this->pictureBox1->Image = img; 
		}
		
		private: System::Void textBox1_TextChanged(System::Object^ sender, System::EventArgs^ e) {
		}
		private: System::Void textBox2_TextChanged(System::Object^ sender, System::EventArgs^ e) {
		}
		private: System::Void textBox3_TextChanged(System::Object^ sender, System::EventArgs^ e) {
		}
		private: System::Void textBox4_TextChanged(System::Object^ sender, System::EventArgs^ e) {
		}
	};
}
\end{verbatim}

\section{Содержание}
	1. Информация о лабораторной работе
	2. Результаты лабораторной работы
	(a) График фигуры до моделирования точек и после 
	(b) График зависимости погрешности вычислений от количества
	испытаний N
	(c) Вывод по лабораторной работе
	(d) Код программы
	3. Содержание
\tableofcontents
	
\clearpage
\end{document}